%%%%%%%%%%%%%%%%%%%%%%%%%%%%%%%%%%%%%%%%%%%%%%%%%%%
% This is the start of a paper by Stephen J Maher %
%            Starting January 2012                %
%%%%%%%%%%%%%%%%%%%%%%%%%%%%%%%%%%%%%%%%%%%%%%%%%%%


\documentclass[11pt,a4paper] {article}

\oddsidemargin -0.0 in
\evensidemargin 0.0in
\textwidth 6.25in
\headheight 0.0in
\topmargin 0.0in
\textheight 9.0in
\evensidemargin 0mm
\pagestyle{headings}
\usepackage{authblk}
\usepackage{amsmath}
\usepackage{amssymb}
\usepackage{amsthm}
\usepackage{delarray}
\usepackage{amsfonts}
\usepackage{cite}
%\usepackage{undertilde}  Remember to uncomment later
\usepackage{bm}
\usepackage{pstricks}
\usepackage{pst-plot}
\usepackage{pst-eps}
\usepackage{pst-grad}
\usepackage{algorithm}
\usepackage{algorithmic}
\usepackage{eurosym}
\usepackage{graphicx}
\usepackage{multirow}
\usepackage{rotating}
% \usepackage{subfigure}
\usepackage{caption}
\usepackage{subcaption}

\usepackage[normalem]{ulem}

\allowdisplaybreaks[1]

\linespread{1.5}

%Adjusting float parameters
\renewcommand{\textfraction}{0.05}
\renewcommand{\topfraction}{0.95}
\renewcommand{\bottomfraction}{0.95}
\renewcommand{\floatpagefraction}{0.5}


\newcommand{\todo}[1]{\par\vspace{2mm} \noindent
\marginpar{to do}
\framebox{\begin{minipage}[c]{0.85\textwidth}
\tt #1 \end{minipage}}\vspace{2mm}\par}
\newcommand{\alignName}[2]{\parbox{\textwidth}{\parbox[l]{0.0001\textwidth}{\noindent(#1)}\parbox[l]{0.967\textwidth}{ #2 }}}

\newcommand{\tocheck}{\marginpar{\emph{check}}}
\newcommand{\reviewed}{\marginpar{reviewed}}
\newcommand{\etal}[1]{\emph{et al.}\cite{#1}}
\newcommand{\etalm}[1]{\emph{et al.}\mbox{\cite{#1}}}
\newcommand{\citem}[1]{\mbox{\cite{#1}}}
\newcommand{\pbspsS}{PBSP-$s$ }
\newcommand{\pbsps}{PBSP-$s$}

\newcommand{\env}{Env}


\newtheorem{mydef}{Definition}[subsection]
\newtheorem{mythm}{Theorem}[subsection]
\newtheorem{mylem}{Lemma}[subsection]
\newtheorem*{myprf}{Proof}

\graphicspath{{./Pictures/}}

%______________________________________________________________________%

\date{}


\begin{document}

\iffalse

What does the teacher scheduling involve?
Input

Sets
- Teachers
- Duties


For each teacher
- Days of work
- Allowable duties
- Number of duties per day
- Piecewise linear cost of number of duties per day

For each duty
- Number of teachers required
- Days the duty is required (possibly every day)

\fi

Let $T$ denote the set of teachers, $D$ the set of duties and $W$ the set of days. The parameters $a_{it}$ are defined to indicate that teacher $t$ works on day $i$. The number of duties that teacher $t$ can perform on day $i$ is denoted by $N_{it}$. Teacher $t$ may not be able to perform all duties of a given day. As such, the parameters $b_{idt}$ are defined to equal 1 is teacher $t$ can perform duty $d$ on day $i$. The variables $x_{idt}$ are defined to equal 1 if teacher $t$ performs duty $d$ on day $i$. Finally, a duty may not be required for each day in $W$. If duty $d$ must be performed on day $i$, then the parameter $p_{id}$ equals the number of staff that are required to perform duty $d$ on day $i$.

We define the mathematical model of teacher duty scheduling as

\begin{aligned}
  \min \quad&
  \text{s.t.} \quad&
  % Duty coverage, days of work, skills required
  &\sum_{t \in T}a_{it}b_{idt}x_{idt} = p_{id} \quad \foral d \in D, \forall i \in W, \\
  %%% Alternatively
  %&\sum_{t \in T}x_{idt} = p_{id} \quad \forall d \in D, \forall i \in W, \\
  %$\sum_{t \in T}\sum_{d \in D}x_{idt} \leq a_{it} \quad \forall i \in W, \\ 
  %$\sum_{t \in T}x_{idt} \leq b_{idt} \quad \forall d \in D, \forall i \in W, \\
  % Maximum number of duties per day
  &\sum_{d \in D}x_{idt} \leq N_{it} \quad \forall t \in T, \forall i \in W, \\
  &x_{idt} \in \{0, 1\}
\end{aligned}


\end{document}
